%% ------------------------------------------------------------------------- %%
\chapter{Fundamentos}
\label{cap:fundamentos}

% Texto texto texto texto texto texto texto texto texto texto texto texto texto
% texto texto texto texto texto texto texto texto texto texto texto texto texto
% texto texto texto texto texto texto texto texto texto texto texto texto texto
% texto texto texto texto texto texto texto texto texto texto texto texto texto
% texto texto texto texto texto texto.

%% ------------------------------------------------------------------------- %%
\section{Recuperação booleana}
\label{sec:recuperacao_booleana}

Com o passar do tempo os sistemas de busca foram se popularizando e aprimorando, com respostas que facilitam a tomada de decisão do usuário, fornecendo um conjunto de opções supostamente mais relevantes. Esse processo é conhecido como Recuperação de Informação. \\

``Recuperação de Informação é encontrar materiais (normalmente documentos) de natureza não-estruturada (normalmente texto) que satisfaçam uma informação necessária em grandes coleções (normalmente armazenadas em computadores).'' (Manning, Raghavan, Schütze, 2008, p.1) \\

Por décadas, utilizou-se sistemas de recuperação de informação em ambientes restritos com usuários específicos utilizando alguma linguagem de consulta. No começo da década de 90, coincidentemente com o advento da Internet, sistemas com consultas em linguagem natural tiveram maior destaque.

Não seria prático executar tais buscas em todos os documentos toda vez que uma informação fosse requerida, tornando-se um processo inviável dependendo do tamanho da coleção. Por exemplo, com o número de páginas na Internet esse processo seria extremamente lento. Portanto, criar um índice dos documentos de antemão permite executar a busca mais rapidamente, ao custo do armazenamento e manutenção dele.

Uma forma de fazer tais buscas é por meio do Modelo de Recuperação Booleana (do inglês, \emph{Boolean Retrieval Model}), o qual permite consultas com expressões booleanas, utilizando termos e operadores lógicos como \emph{AND}, \emph{OR} e \emph{NOT} para conectá-los. Com esse modelo, o usuário consegue encontrar precisamente que documentos dentro da coleção satisfazem a sua busca.

Cada documento possui um conjunto de termos nele presente, que o distingue de outros, e para cada termo que se busca, existem documentos em que esse está presente. Assim, a estrutura de dados mais intuitiva a priori é a matriz de incidência, na qual cada posição pode ser 1 ou 0, se o documento contém o termo ou não. Portanto, essa matriz é criada levando-se em conta todos os termos presentes na coleção. Para saber quais documentos contêm certo termo, consulta-se a linha da matriz correspondente ao termo.

Para uma determinada coleção de documentos, o conjunto de termos presentes é chamado de vocabulário ou léxico e a estrutura de dados que guarda o vocabulário é chamada de dicionário.

Existe aqui uma diferença entre a informação necessária, que é aquilo que o usuário visa encontrar de fato, e consulta, que é o que o usuário fornece ao sistema. Um documento é dito relevante se contém informações que o usuário julgue compatíveis com o que lhe é necessário. Podemos medir a eficiência do sistema quanto a uma busca de duas formas: sua precisão ou quantos dos documentos retornados são relevantes; e seu \emph{recall} ou que parcela dos documentos relevantes do sistema foi retornada.

Tendo uma consulta de dois termos ``$t_{1}$ \emph{AND} $t_{2}$'', o modelo de recuperação booleana considera as linhas da matriz de incidência como dois números binários e aplica o operador. Portanto nesse modelo, temos um baixo \emph{recall}, pois o sistema filtra os documentos a partir da consulta, não tendo flexibilidade para incluir outros que poderiam ser relevantes. É um modelo muito limitado, tendo em vista as necessidades dos usuários.

Em geral, a matriz de incidência dos termos é extremamente esparsa. Se tivermos documentos de 1000 palavras, mas um dicionário com $10^{6}$ termos, teremos cada linha da matriz com no máximo 0,1\% das posições com 1. 

Entretanto, existe uma estrutura mais compacta chamada Lista Invertida (do inglês, \emph{inverted index}), que indica para cada termo do dicionário os documentos em que esse está presente. Cada registro nessa lista é chamado de postagem (do inglês, \emph{posting}). Para cada termo temos uma lista de postagens. Para gerar o índice temos os seguintes passos:
\begin{enumerate}
\item Reunir os documentos que farão parte do índice.
\item Criar \emph{tokens} para as palavras criando uma lista para cada documento.
\item Normalizar estes \emph{tokens} por meio de um processo linguístico. Os \emph{tokens} normalizados serão os termos indexados (o dicionário).
\item Atribuir uma identificação (por exemplo, número inteiro em série) para cada documento e atribuí-los às listas dos termos que neles aparecem.
\end{enumerate}
Depois, ordena-se o dicionário em ordem alfabética. Ocorrências repetidas de um mesmo termo em um documento são mescladas e entradas repetidas de um mesmo termo no dicionário são agrupadas. O produto final é composto pelo dicionário de termos e suas postagens. O dicionário pode ser armazenado na memória, dependendo de seu tamanho, enquanto as postagens serão lidas do disco.

Essa estrutura de dados pode ser usada para guardar estatísticas, como o número de documentos em que um termo aparece (que é também o comprimento da lista de postagens). Essas estatísticas podem ser usadas para ranquear os resultados de uma busca de forma mais eficiente.

As listas de postagens podem ser feitas com diferentes estruturas de dados, como vetores e listas ligadas. Vetores de tamanho fixo são pouco eficientes, pois podemos ter listas de postagens de tamanhos muito discrepantes, o que resultaria em desperdício de espaço. Com vetores de tamanho variável, perde-se tempo apenas no redimensionamento, logo, se o índice não for muito atualizado, esse pode ser útil. Por sua vez, listas ligadas precisam de mais espaço por causa dos ponteiros.

No modelo de recuperação booleana, tendo essas listas ordenadas pelas identificações dos documentos, é fácil de processar buscas. No caso do operador \emph{AND}, basta recuperar as listas de dois dos termos e selecionar os termos presentes nas duas listas. Já que as listas estão ordenadas podemos avançar intercaladamente entre elas. Sendo $n_{i}$ o tamanho de cada uma das listas, essa operação tem $O(n_{1}+n_{2})$ comparações. Para processar a consulta inteira, pode-se fazer uma operação de cada vez. Assim, a complexidade da busca para uma consulta genérica é $\Theta(N)$, sendo $N$ o tamanho do dicionário, que na prática é uma constante imensa.

Uma forma de otimizar o processamento da consulta é mudar a ordem em que as operações são feitas, ordenando os termos da consulta pelo tamanho de suas listas de postagens. Nenhum resultado parcial será maior do que a menor das listas utilizadas até ali, logo, começando pelas operações com as listas menores, o tamanho dos resultados parciais será sempre menor ou igual a menor das listas utilizadas na consulta.

Existem outros modelos de recuperação de informação, como modelos de recuperação ranqueada (do inglês, \emph{ranked retrieval models}), no qual a consulta submetida pelo usuário assume formato livre e o sistema deve responder a essas consultas. Um exemplo é o modelo de espaço vetorial contrastando com o uso de operadores.

%% ------------------------------------------------------------------------- %%
\section{Vocabulário de termos e listas de postagens}
\label{sec:lista_postagens}

\subsection{Decodificação de sequências de caracteres e delimitação de documento}
%
%\subsubsection{Extrair sequências de caracteres de um documento}

Os documentos digitais são entradas para um indexador e são tipicamente bytes de um arquivo ou de uma página web e, sendo assim, a primeira etapa do processo de indexação é converter esses bytes para uma sequência linear de caracteres. Mas antes é preciso determinar a codificação correta do documento. Determinada a codificação, é feita a conversão de bytes para caracteres. 

\subsection{Escolhendo a unidade de documento}

Para uma coleção de livros, normalmente seria má ideia indexar um livro inteiro como um documento, pois a busca por ``brinquedos chineses'' pode retornar um livro que menciona China no primeiro capítulo e brinquedos no último, o que não torna o livro relevante. Ao invés disso, podemos indexar cada capítulo ou parágrafo ou até mesmo sentenças individuais como documentos. Com isso, uma vez que os documentos são menores, será muito mais fácil para o usuário encontrar passagens relevantes dentro do documento. Torna-se claro que há uma recompensa com isso em termos de \emph{precision/recall}. Se as unidades forem muito pequenas, é provável que se perca passagens importantes, porque os termos foram distribuídos sobre vários mini-documentos, enquanto que se as unidades forem muito grandes, há a tendência de obtermos correspondências espúrias e a informação relevante torna-se difícil de ser encontrada pelo usuário.

\subsection{Determinando o vocabulário de termos}

\subsubsection{Tokenização}

Tokenização é a tarefa de dividir uma sequência de caracteres em pedaços menores, chamados de \emph{tokens}, podendo eventualmente descartar certos caracteres, como pontuação. Um \emph{token} é uma instância de cadeia de caracteres em um documento que é agrupada como uma unidade semântica útil para processamento. Segue abaixo um exemplo de tokenização:

Entrada: Friends, Romans, Countrymen, lend me your ears

Saída: [Friends] [Romans] [Countrymen] [lend] [me] [your] [ears]

Um tipo é a classe de todos os \emph{tokens} contendo a mesma sequência de caracteres. Um termo é um tipo (talvez normalizado) que é incluso no dicionário do sistema de RI. Múltiplos \emph{tokens} que são reunidos em conjunto via normalização são indexados como um termo sob forma normalizada. Por exemplo, se o documento a ser indexado é \emph{``to sleep perchance to dream''}, então há 5 \emph{tokens}, mas apenas 4 tipos (uma vez que há 2 instâncias de \emph{to}). Entretanto, se \emph{to} for omitido do índice (como uma \emph{stop word}), então haverá somente 3 termos: \emph{sleep}, \emph{perchance}, e \emph{dream}.

A principal questão na fase de tokenização é qual convenção usar? Corta-se nos espaços em branco ou não, remove-se os caracteres de pontuação, o que fazer em casos com hífen, etc. Cada idioma apresenta questões particulares que devem ser consideradas e alguns até mesmo fazem da tokenização uma tarefa muito complexa. Todos os métodos cometem erros algumas vezes, e então nunca há a garantia de uma tokenização única e consistente.

\subsubsection{Eliminando termos comuns: \emph{stop words}}

\emph{Stop words} são \emph{tokens} que não são indexados e, portanto, não são termos do vocabulário. Tais \emph{tokens} são muito comuns e parecem ser de pouco valor para tornar um documento relevante ao usuário (e.g. artigos, pronomes, preposições, entre outros).

Alguns sistemas de RI excluem do vocabulário as \emph{stop words} e para isso, determina-se os termos mais frequentes da coleção, podendo ser necessário estudar a relação da semântica do termo com o domínio de indexação. Entretanto, nem sempre é razoável eliminar as \emph{stop words}, pois podem ser úteis em casos de consultas de frases (e.g. ``presidente do Brasil'' é mais preciso do que ``presidente'' \emph{AND} ``Brasil'').

A princípio, o custo de inclusão de \emph{stop words} não é tão grande, tanto em armazenamento quanto em processamento, e em geral, ferramentas de busca da Web nem as utilizam.

\subsubsection{Normalização (classes de equivalência de termos)}

Nesta etapa é que são gerados os termos que aparecerão no dicionário. Após ``quebrar'' a coleção e a consulta em \emph{tokens}, casos de correspondência exata entre os \emph{tokens} não são sempre verdade, mas há casos onde é desejável agrupar \emph{tokens} semelhantes por classes de equivalência ou por algum outro método de agrupamento.

Agrupá-los por classes de equivalência tem suas vantagens em termos de desempenho e significa aplicar regras de eliminação de hífens, diacríticos, redução para o minúsculo (e entre outros) e o resultado dessas regras geram o nome das classes. Em outras palavras, implica converter os \emph{tokens} para a forma canônica. Outras questões particulares de cada idioma devem ser levadas em conta tanto na etapa de tokenização quanto de normalização e muitas vezes é útil a aplicação de classificadores de idioma para a melhor seleção das regras em ambas as etapas.

\subsubsection{Stemização e Lematização}

Ambos termos se referem a algoritmos que permitem a redução de palavras flexionadas ou derivadas à sua forma base ou raiz, permitindo que variantes gramaticais de uma palavra sejam agrupadas como uma só unidade. A normalização continua sendo o objetivo aqui.

Stemização geralmente se refere a uma heurística bruta de cortes de afixos (isto é, prefixos e sufixos) para encontrar a raiz da palavra, o que torna sua implementação mais fácil, rápida e satisfatória para muitas aplicações, permitindo alto \emph{recall}, mas baixa precisão.

Lematização faz uso de vocabulário e análise morfológica para identificar a classe gramatical da palavra, e assim determinar o lema da palavra, que é a sua forma de dicionário (forma raiz). Diferente da stemização, a lematização distingue o contexto das palavras, selecionando seu radical apropriado.

Exemplo: %(https://en.wikipedia.org/wiki/Lemmatisation)

\begin{enumerate}
\item A palavra ``\emph{better}'' possui ``\emph{good}'' como lema. Essa associação é perdida na stemização, mas obtida na lematização, pois requer um dicionário.

\item A palavra ``\emph{walk}'' é a forma base da palavra ``\emph{walking}'', e portanto isso é conseguido tanto pela stemização como pela lematização.
\end{enumerate}

\subsection{Intersecção de listas de postagens mais rápida via \emph{skip pointers}}

Se o índice for relativamente estático, é possível implementar intersecção de listas de postagens mais eficiente usando \emph{skip pointers}. Tradicionalmente, dadas duas listas de tamanho $m$ e $n$, a operação de intersecção consome tempo $O(m+n)$, pois percorre simultaneamente ambas as listas. Já com \emph{skip pointers}, é possível reduzir essa complexidade para tempo sublinear. Alocando-se uma quantidade arbitrária e heuristicamente razoável de ponteiros, ganha-se tempo ao evitar ter de comparar postagens menores de uma das listas com o elemento de comparação da outra lista, pois não apareceriam de qualquer maneira no resultado da operação.

\subsection{Postagens posicionais e consultas de frase}

Na maioria das situações, os usuários irão expressar sua consultas por meio de frases, e assim é desejável que a ferramenta de busca forneça suporte adequado e eficiente para consultas mais complexas que envolvam múltiplos termos, onde a proximidade entre eles é um fator de relevância adicional.

\subsubsection{Índice de dupla palavra}

Após a tokenização, é necessário realizar classificação linguística das palavras (e.g. susbtantivo, verbo, preposição, entre outros) por meio do uso de ferramentas computacionais próprias. Com isso, torna-se possível indexar os documentos como duplas palavras estendidas na forma N X* N, onde N representa uma palavra ``relevante'' e X, uma palavra funcional de baixa relevância, como preposição, artigo ou conjunção. Assim, as entradas do dicionário passam a assumir a forma N X* N, tornando possível a realização de consultas de frase por parte do usuário final. Tal conceito pode ser estendido para mais de duas palavras, e assim denominando-se índice de frases. Índices de frases em geral expandem muito o vocabulário, e não impedem que falsos positivos ocorram. Quando o índice de dupla palavra é usado, deve-se manter também um índice de termos únicos, para que buscas por termos individuais possam ser realizadas.
 
Exemplo de formato de dupla palavra:

\begin{table}[!h]
	\centering
	\caption{Formato de dupla palavra.}
	\begin{tabular}{|c|c|c|c|}
		\hline
		\emph{renegotiation} & \emph{of} & \emph{the} & \emph{constitution} \\
		\hline
		N & X & X & N\\
		\hline 
		\end{tabular}
\end{table}

\subsubsection{Índices posicionais}

Índices de dupla palavra não são o padrão, por razões de espaço requerido. O mais usado para dar suporte a consultas de frase e de proximidade é o índice posicional: \\

%----------------------------------------------------
%to , 993427:
%
%< 1, 6: < 7, 18, 33, 72, 86, 231 > ;
%
%\hspace{0.4cm}2, 5: < 1, 17, 74, 222, 255 >; ...>\\
%----------------------------------------------------
to , 993427: < 1, 6: < 7, 18, 33, 72, 86, 231 > ;\hspace{0.4cm}2, 5: < 1, 17, 74, 222, 255 >; ...>\\

No exemplo acima, o termo to tem valor df 993427 (nº de ocorrências do termo na coleção). A primeira postagem são ocorrências do termo no documento 1, e seu valor tf vale 6 (número de ocorrências do termo no documento). A lista em seguida são as posições de ocorrência do termo no documento. Porém, os índices posicionais tendem a exigir mais requisitos de espaço e a ser menos eficiente do que os outros índices.

%% ------------------------------------------------------------------------- %%
\section{Dicionários e recuperação tolerante}
\label{sec:dicionarios}

  \emph{Hashes} e árvores são estruturas de dados extremamente importantes na realização de consulta, pois ajudam a determinar se cada termo da consulta existe no vocabulário. No primeiro, cada termo do vocabulário(chave) é mapeado para um inteiro com espaço suficiente para que as colisões de \emph{hash} sejam improváveis. No segundo, a árvore de busca mais conhecida é a árvore binária, na qual cada nó interno tem dois filhos. A busca de um termo começa na raiz da árvore. Cada nó interno (incluindo a raiz) representa um teste binário, baseado nesse resultado a busca prossegue para uma das sub-árvores abaixo desse nó. Contudo, um dos problemas enfrentados pelas árvores binárias é o rebalanceamento, portanto, para suavizar esse problema são usadas árvores B, as quais são árvores de busca em que cada nó interno tem um número de filhos no intervalo [a-b].
   
  Um ponto negativo dos \emph{hashes} é que em uma ferramente (tal como a Web) cujo tamanho do vocabulário permanece aumentando, uma função de \emph{hash} projetada para necessidade atual pode não ser suficiente daqui algum tempo.
  
  Há consultas conhecidas como ``consultas curingas'', as quais são, normalmente, usadas pelo usuário quando ele não tem certeza de como se escreve um termo da consulta ou está ciente das várias formas de escrita de um termo, então procura documentos contendo qualquer uma dessas variações. Para realizar essas consultas curingas, o usuário precisa digitar o caracter *, como nos exemplos abaixo: \\
    
\indent S*dney $\rightarrow$ Sydney ou Sidney \\
\indent *artoze $\rightarrow$ Catorze ou Quatorze \\

  Para manipular consultas em que há um único símbolo *, tal como ``S*dney'', costuma-se usar duas árvores-B, árvore-B normal e árvore-B reversa, e depois pegar a intersecção de ambas.    

  Além disso, algumas técnicas robustas para consultas curingas e correções ortográficas são utilizadas tais como os índices permuterm e os índices k-grama. 
    
  O índice permuterm consiste em marcar o final do termo com o caracter \$ e então buscar no dicionário todas as rotações com esse índice. Um exemplo do índice permuterm para a palavra carro é:  carro\$, arro\$c, rro\$ca, ro\$car, o\$carr. Uma desvantagem do índice permuterm é que o seu dicionário se torna bastante grande, já que inclui todas as rotações de cada termo. 
    
  Dessa forma, uma outra técnica seria o índice k-grama. Um k-grama é uma sequência de k caracteres. Além disso, usa-se um caractere especial \$ para denotar o começo ou o fim de um termo, de modo que o conjunto completo de 3-gramas gerado para o termo carro é: \$ca, arr, rro, ro\$. Em um índice k-grama, o dicionário contém todos os k-gramas que ocorrem em qualquer termo do  vocabulário. Cada lista de postagem aponta de um k-grama para todos os termos de vocabulário que contém esse k-grama.

%% ------------------------------------------------------------------------- %%
\section{Contrução do índice}
\label{sec:construcao_indice}

Para criar uma lista invertida para uma coleção de documentos, primeiro cria-se um lista de pares termo-\emph{docID}, ordenando-a pelos termos e mantendo a sequência da identificação dos documentos. Com isso, podemos criar estatísticas como a frequência de documentos e a do termo. Para coleções pequenas, a memória pode ser suficiente para esta tarefa. Para coleções maiores, no entanto, será necessário também o uso do disco.

Para a discussão que se segue, vale lembrar alguns pontos:
\begin{itemize}
\item O acesso à memória é muito mais rápido do que acessar dados no disco.
\item A movimentação da \emph{disk head} é em geral muito lenta, e durante esse processo não há transferência de dados. Logo, é melhor transferir dados que estejam agrupados no menor número de blocos possível.
\item Os sistemas operacionais leem blocos inteiros do disco, logo, ler 1 byte ou um bloco inteiro representa a mesma operação.
\item A leitura do disco não é executado pelo processador, logo, esse está livre durante a leitura. Uma forma de se aproveitar esse fato é gravar dados comprimidos no disco e ter um algoritmo de descompressão eficiente, de forma que a leitura mais o tempo de descompressão dos dados seja menor que a leitura dos dados descomprimidos.
\end{itemize}

Operações com strings podem ser custosas, portanto, é preferível o uso de \emph{termIDs} ao invés dos termos em si. Estas identificações podem ser geradas durante a criação da lista invertida ou em uma etapa anterior, criando primeiramente o dicionário.

Se cada \emph{termIDs} ou \emph{docIDs} ocupar 4 bytes, uma coleção de 100 milhões de tokens somaria 0,8 GB. A ordenação destes pares tomará ainda mais espaço, dependendo do algoritmo que for utilizado. A memória sendo insuficiente, torna-se necessário o uso do disco. Além disso, o uso de algoritmos de ordenação externa que minimizem o número de acessos aleatórios ao disco, já que buscas em blocos sequenciais são mais rápidas. Um exemplo desse tipo de algoritmo é o \emph{blocked sort-based indexing algorithm} (BSBI). 

Esse algoritmo cria blocos de mesmo tamanho contendo os pares \emph{termID-docID}, ordena cada bloco em memória guardando os resultados parciais no disco e, por fim, mescla os blocos na lista final. Os blocos devem caber na memória de forma que permitam a sua ordenação sem o uso do disco. Após a ordenação, cada bloco é uma lista invertida de um pedaço da coleção. Assim, para mesclar as listas, mantém-se aberto um buffer de leitura para cada bloco e um buffer de escrita, onde será escrito a lista resultante. Escolhemos o menor \emph{termID} que não está na lista final e mesclamos as listas de postagens de todos os blocos; essa será a lista de postagens desse \emph{termID} na lista inversa resultante. 

A complexidade do algoritmo é equivalente à uma ordenação. Sendo $T$ um limitante superior proporcional a quantidade de pares \emph{termID-docID}, temos que esta complexidade é $\Theta( T \cdot log T )$. Na prática, os passos de análise dos documentos e mesclagem da lista final influenciam o tempo total da execução do algoritmo.

O algoritmo BSBI requer que a estrutura de dados que mapeia termos aos seus respectivos \emph{termIDs} seja guardada na memória. Para coleções muito grandes isso se torna impraticável. Uma alternativa é o algoritmo \emph{single-pass in-memory indexing} (SPIMI) que usa os próprios termos ao invés de seus \emph{termIDs}, criando um dicionário para cada bloco e guardando-o no disco. 

Primeiro, é necessário parsear os documentos em pares de termos e \emph{docIDs}. Para cada um destes pares, verifica-se no dicionário se o termo já está presente e resgata-se a sua lista de postagens. Assim, cada postagem é adicionada a lista de postagens individualmente, diferentemente do algoritmo BSBI que primeiro recolhe todos os pares para depois organizá-los. Isso torna o algoritmo SPIMI mais rápido por não precisar fazer este último passo e, além disso, não precisa guardar os \emph{termIDs} por manter uma relação entre um termo e a sua lista de postagens, o que permite utilizar melhor a memória com blocos maiores. Quando a memória estiver cheia, escreve-se a lista em bloco no disco, apenas tomando o cuidado de ordenar os termos do dicionário, o que facilita a mesclagem das listas, por permitir encontrar um termo com uma busca linear em qualquer dicionário. Por fim, mescla-se as listas de todos os blocos. 

O algoritmo SPIMI também permite a compressão das listas de postagens e do dicionário, o que permite processar blocos maiores de cada vez e economiza espaço do disco. Por não precisarmos ordenar uma lista de postagens, o algoritmo tem complexidade $\Theta( T )$, ou seja, não tem operações que sejam mais do que lineares no tamanho da coleção.
Mas, e se uma coleção for muito grande para ser indexada em um único computador? Neste caso, são utilizados algoritmos de indexação distribuída, que criam um índice armazenado em mais de uma máquina. Cada máquina pode conter um pedaço de cada lista de postagens ou, para um conjunto de termos, suas listas de postagens. Assim, podemos usar o processamento distribuído de um \emph{cluster}, distribuindo parcelas da coleção de documentos a ser indexada para cada máquina.

Um problema com esse método é a atribuição de \emph{termIDs} para os termos, já que cada unidade de processamento terá uma tabela própria. Um modo de contornar isso é preprocessar uma tabela de \emph{termIDs} para os termos mais recorrentes que será distribuída para todas as máquinas e, para os termos menos frequentes, atribuir as listas de postagens aos termos em si. Com isso, cada máquina cria arquivos intermediários que guardam um conjunto de pares de valores para as listas de postagens. Por exemplo, para cada máquina teríamos um arquivos com os pares correspondentes aos termos que começam com a letra `a' até a letra `g', outro arquivos para os termos que começam com a letra `h' até a letra `p', e assim por diante. Depois estes arquivos serão mesclados de forma que uma máquina contenha todos os pares para um destes segmentos de termos. Digamos, uma máquina recebe os arquivos que contém os pares com os termos de `a' até `g', mescla estas listas e mantém as listas de postagens resultantes.

Tanto os algoritmos BSBI e SPIMI quanto a indexação distribuída trabalham, a princípio, com coleções estáticas, ou seja, cada documento será indexado apenas uma vez. Mas coleções que se modificam ao longo do tempo precisarão de atualizações eventualmente. Se as mudanças forem pouco frequentes, uma opção seria reindexar a coleção inteira, mantendo uma versão estável para as buscas. Se for necessário adicionar versões mais recentes dos documentos de forma rápida, no entanto, isto deixa de ser uma opção. 

Uma lista auxiliar seria útil neste caso. Manteríamos a lista invertida original em disco e a lista auxiliar com as entradas novas, em memória. Ao invés de atualizarmos frequentemente a lista, podemos concentrar as mudanças nesta lista auxiliar e fazer acessos ao disco de uma vez só. Com este esquema, uma busca percorre ambas as listas para obter seu resultado. Entradas novas são adicionadas a lista auxiliar e entradas que foram deletadas são mantidas em um vetor de validação ou estrutura similar que possa ser usada para filtrar os resultados. Assim, para atualizar um documento, basta deletar suas entradas da lista original (ou seja, adicioná-las ao vetor de deleções) e adicionar as novas entradas na lista auxiliar. Quando a lista auxiliar estiver muito grande, enfim, mesclamos as duas listas. Se tivéssemos cada lista de postagens em um arquivo separado isto seria fácil, bastaríamos mesclar as mudanças pertinentes a este arquivo. Mas em geral este não é o caso, já que os sistemas operacionais não lidam bem com um número muito grande de arquivos. 

Para que o número de arquivos não cresça demasiadamente, uma possibilidade é usar o algoritmo \emph{logarithmic merge}, que utiliza um conjunto de listas guardadas em memória cada uma com o dobro do tamanho da anterior. O tamanho da primeira lista é igual ao tamanho máximo $n$ da lista auxiliar e toda vez que um nível estiver cheio, mesclamos com o próximo. Para um conjunto de postagens com $T$ elementos a serem processados, teremos $log( T/n )$ níveis. Por exemplo, se tivéssemos n postagens, apenas a lista auxiliar seria necessária. Com isso, a inserção de uma postagem ao índice terá complexidade $\Theta( log( T/n ))$, pois passará no máximo por todos os níveis. A indexação de todas as postagens, por sua vez, terá complexidade $\Theta( T log(T/n ))$.

O problema desse método é que uma busca terá que percorrer todos os níveis mesclando os resultados e não só as duas listas que tínhamos anteriormente. Pode ser preferível, portanto, reconstruir o índice de tempos em tempos, dependendo da aplicação.

%% ------------------------------------------------------------------------- %%
\section{Pontuação, ponderação do termo e modelo do espaço vetorial}
\label{sec:espaco_vetorial}

\subsection{Atribuir pontuação a um documento}

Para ranquear os resultados de uma busca é necessário algum sistema de pontuação. Um do-cumento que menciona um termo da consulta mais vezes é mais interessante, logo, deve receber uma pontuação maior. Essa pontuação de um termo $t$ em um documento $d$ será chamado de peso de $t$ em $d$. Pensando em consultas de texto livre, que é uma forma mais comum na web, seria fácil computar a pontuação de um documento, sendo esta soma dos pesos dos termos da consulta presentes no documento. Segundo Manning, Raghavan e Schütze (2008), a forma mais natural de atribuir esse peso é pela frequência do termo, denotada $tf_{t,d}$, mas pode-se usar qualquer função de ponderação. Essa forma é conhecida como modelo saco-de-palavras (do inglês, \emph{bag of words model}) que leva em consideração apenas o número de ocorrências dos termos, mas não a ordem destes. Mesmo que a ordem das palavras divirja entre dois documentos, se as frequências dos termos forem similares, esses são considerados similares.

Alguns termos podem ser muito frequentes na coleção, logo, não terão muita influência no ranqueamento dos documentos, sendo necessária algum tipo de atenuação. Utilizando a frequência do termo na coleção, um termo muito presente em poucos documentos e um termo pouco presente em muitos documentos teriam a mesma frequência, então, a atenuação não surtiria efeito. Portanto, é comum usar a frequência de documentos ($df_{t}$), ou seja, o número de documentos da coleção em que o termo está presente. Assim, definimos frequência inversa do documento ($idf$):

$$idf_{t} = log \frac{N}{df_{t}}$$

Um termo que está em todos os documentos terá $idf$ = 0. Quanto mais raro o termo na coleção, maior será o seu $idf$. Multiplicando $tf_{t,d}$ por $idf_{t}$ temos um balanceamento entre a presença de um termo em um documento e a sua raridade na coleção. Um documento que diz muito sobre um assunto que aparece pouco na coleção deve ser mais relevante para este assunto. Definimos então o balanceamento $tf\textnormal{-}idf$:

$$tf\textnormal{-}idf_{t,d} = tf_{t,d} \cdot idf_{t}$$

Trabalhando com o documento como um vetor de pesos para os termos do dicionário, temos a pontuação de um documento $d$ em relação à uma consulta $q$:

\begin{displaymath}
score(q,d) = \sum_{t \in q} tf\textnormal{-}idf_{t,d}
\end{displaymath}

Esta representação de documentos como vetores em um espaço vetorial é conhecida como modelo vetorial (do inglês, \emph{vector space model}) e pode ser usado em outras operações de recuperação de informação. Além disso, outras funções de ponderação podem ser usadas no lugar de $tf\textnormal{-}idf$. Para cada documento $d$ temos, então, um vetor $\overline{V}(d)$ com componentes sendo as ponderações. Assim, o espaço vetorial terá um eixo para cada termo.

Dois documentos serão similares se tiverem vetores com componentes proporcionalmente parecidas, mesmo que as componentes de um documento sejam maiores do que a do outro. Para computar esta similaridade, é comum o uso do cosseno entre os dois vetores:

$$sim(d_{1},d_{2}) = \frac{\overline{V}(d_{1})\cdot \overline{V}(d_{2})}{|\overline{V}(d_{1})|\cdot|\overline{V}(d_{2})|}$$

Temos a divisão do produto vetorial entre os vetores sobre o produto de seus comprimentos euclidianos. Esta divisão é necessária para normalizar os resultados. Se os vetores já estiverem normalizados, apenas o produto vetorial é necessário.

A coleção será representada pelo conjunto destes vetores: uma matriz MxN, sendo M o número de termos do dicionário e N o número de documentos. Da mesma forma que um documento, uma consulta pode ser representada por um vetor. Sendo assim, a pontuação de um documento em relação a uma consulta será:

$$score(q,d) = \frac{\overline{V}(q) \cdot \overline{V}(d)}{|\overline{V}(q)|\cdot|\overline{V}(d)|}$$

Além do uso do cosseno para o cálculo da similaridade, outras funções podem ser usadas. Busca-se, hoje, determinar de maneira mais formal a influência destas funções em diferentes domínios.

\subsection{Índices paramétricos e de zonas}

Buscas poderão depender de outras informações que estão além do corpo de um documento. Como alternativa, Moura, Pereira e Campos (2002) propõem o uso de metadados como uma solução adequada para promover uma recuperação de recursos na Web mais eficiente e precisa, já que permite um melhor agrupamento de recursos digitais. Por exemplo, poderíamos querer resgatar documentos que foram criados antes de uma data ou por um autor específico. Guardando estas informações (data de publicação e autor, no exemplo) como metadados dos documentos, podemos recuperá-los com precisão. Manning, Raghavan e Schütze (2008) definem estas representações das informações nos metados como campos. Para a indexação destes campos, os autores dão como possibilidade o uso de índices paramétricos (do inglês, \emph{parametric indexes}). Nestes, para cada campo é construída uma lista invertida com suas postagens. Assim, fazemos buscas separadas e mesclamos os resultados.

Os campos de metadados podem ter em um domínio de valores ordenáveis, como datas, por exemplo. Assim, o dicionário deste campo pode ser implementado em uma estrutura que facilite a navegação por ela, como uma árvore binária.

Além dos campos, uma busca pode ser feita em partes específicas de um documento, como o título ou um resumo. Para tal, utiliza-se zonas, similares aos campos, mas que assumem textos de tamanho arbitrário. Para cada zona podemos construir um índice, cujo dicionário é constituído dos termos presentes nesta zona. Outra alternativa é codificar a zona em que um termo aparece e fazermos um índice único, o que reduz o espaço ocupado pelo dicionário.

Até aqui utilizamos os índices paramétricos para fazermos buscas em que um termo está ou não presente em uma zona. Podemos, então, ponderar as zonas e atribuir uma pontuação para cada documento. Por exemplo, talvez seja uma informação mais relevante para o usuário se um termo aparece no corpo de um documento, enquanto o título pode não ser importante. Se tivermos $n$ zonas, e atribuirmos um peso $p_{i}$ para cada zona, podemos normatizar a soma dos pesos para que:

\begin{displaymath}
	\sum_{i=1}^{n} p_{i} = 1
\end{displaymath}

Dado uma query $q$ e um documento $d$, podemos atribuir uma pontuação $s_{i}$ para cada zona do documento. A pontuação total do documento é, portanto, a combinação linear dos pesos e as pontuações das zonas:

\begin{equation}
	score(d,q)= \sum_{i=1}^{n} p_{i} s_{i}
	 \label{score}
\end{equation}

Um problema surge, naturalmente: como determinar os pesos $p_{i}$? Eles podem ser determinados pelo próprio usuário ou ``aprendidos'' através de exemplos escolhidos, ou seja, o aprendizado de máquina. Os autores definem os exemplos como tuplas que contém uma query $q$, um documento $d$ e um julgamento da relevância, que pode ser tão simples quanto ``é relevante'' e ``não é relevante'', mas pode ter outras nuâncias. Através destes exemplos, a máquina gera pesos $p_{i}$ que tenham resultados que se aproximem dos julgamentos de relevância que foram fornecidos. Isto pode ser escrito como um problema de otimização.

No aplicativo \emph{LookingFor} desenvolvido neste Trabalho de Conclusão de Curso, exploramos o primeiro caso: o usuário determina os pesos para cada consulta. No entanto, nenhum parâmetro textual é fornecido. O sistema utiliza parâmetros preestabelecidos para gerar as pontuações. Ou seja, o usuário influencia os pesos $p_{i}$ mas não as pontuações $s_{i}$.