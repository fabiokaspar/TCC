%% ------------------------------------------------------------------------- %%
\chapter{Introdução}
\label{cap:introducao}

Com o grande volume de informações que a Web nos dá acesso, é inevitável que surgissem formas de recuperar informações em largas coleções de documentos. Não só isso, mas também ranquer os resultados. Isto é essencial, já que o número de respostas para uma busca pode ser muito grande. Segundo Manning, Raghavan e Schütze (2008), a área de Recuperação de Informação (RI) está se tornando rapidamente a forma mais utilizada de acesso a informação, pois permite a recuperação rápida de referências a determinados termos. Isto possibilita aos usuários fazerem buscas de forma não estruturada. 

Os autores definem a relevância dos resultados como sendo subjetiva ao usuário, por ser a percepção deste sobre as informações contidas no documento e se estas são de relevância à sua necessidade. Para Crippa e Rodrigues (2011), é justamente esta subjetividade que dificulta alcançar os objetos da área de fornecer o acesso rápido e eficaz às informações relevantes.

Uma das formas de se buscar resultados mais relevantes é guardar outros dados que permitam criar formas de ranquear estes resultados, como a frequência dos termos em um documento, a raridade de um termo na coleção e a quantidade de acessos de uma página Web.

Alguns sites, além disso, nos permitem fazer buscas por informações que estão além do corpo dos documentos. Por exemplo, buscando por sites de restaurantes, um usuário pode achar mais relevantes os resultados com preços mais baixos ou que são mais próximos de sua residência. 

A forma como essa busca é feita pelo usuário varia com a interface apresentada ao usuário. A busca do Google, por exemplo, fornece apenas o campo de texto, permitindo ao usuário fazer buscas nos campos dos documentos já definidos no sistema. Sites como o Guia Folha e o Kekanto utilizam um esquema de filtros para ajudar o usuário a encontrar os resultados mais relevantes. Por exemplo, pode-se filtrar os restaurantes com gastos médios entre R\$75 e R\$95 e que se situam no bairro do Butantã, mas não temos uma visão geral de quais são os estabelecimentos mais próximos e com preços mais baratos, independente da faixa de preço e localização. Se o usuário não tem certeza de quanto quer gastar e até aonde se propõe a ir, precisa trocar os filtros, possivelmente realizando várias buscas até encontrar um resultado satisfatório.

