%% ------------------------------------------------------------------------- %%
\chapter{Bibliografia}
\label{cap:bibliografia}

\begin{enumerate}
\item MANNING, Christopher D.; RAGHAVAN, Prabhakar; SCHÜTZE, Hinrich. Introduction to Information Retrieval, Cambridge University Press. 2008. Available from: $<$http://nlp.stanford.edu/IR-book/pdf/irbookonlinereading.pdf$>$

\item MEIRELES, Magali Rezende Gouvêa; ALMEIDA, Paulo Eduardo Maciel de; SILVA, Ana Carolina Milagres Resende. Recuperação de informação no ambiente acadêmico: georreferenciamento dos dados dos estudantes do Instituto de Educação Continuada da PUC Minas. Perspect. ciênc. inf.,  Belo Horizonte ,  v. 14, n. 3, p. 61-74, Dec.  2009 .   Available from: $<$http://www.scielo.br/scielo.php?script=sci\_arttext\&pid=S1413-99362009000300005\&lng=en\&nrm=iso$>$. Access on 18 Oct. 2015.

% \item ESCUDEIRO, Nuno Filipe; JORGE, Alípio Mário. Satisfying Information Needs on the Web: a Survey of Web Information Retrieval. Tékhne,  Barcelos ,  n. 9, jun.  2008 .   Disponível em $<$http://www.scielo.mec.pt/scielo.php?script=sci_arttext&pid=S1645-99112008000100018&lng=pt&nrm=iso$$>$$. acessos em  18  out.  2015.

% \item TEIXEIRA, Cenidalva Miranda de Sousa; SCHIEL, Ulrich. A INTERNET E SEU IMPACTO NOS PROCESSOS DE RECUPERAÇÃO DA INFORMAÇÃO. Ci. Inf.,  Brasilia ,  v. 26, n. 1, p. , Jan.  1997 .   Available from $<$http://www.scielo.br/scielo.php?script=sci_arttext&pid=S0100-19651997000100009&lng=en&nrm=iso$>$. access on  18  Oct.  2015.  http://dx.doi.org/10.1590/S0100-19651997000100009.

% \item MOURA, Ana Maria de Carvalho; PEREIRA, Genelice da Costa; CAMPOS, María Luiza Machado. A metadata approach to manage and organize electronic documents and collections on the web. J. Braz. Comp. Soc.,  Campinas ,  v. 8, n. 1, p. 16-31, July  2002 .   Available from $<$http://www.scielo.br/scielo.php?script=sci_arttext&pid=S0104-65002002000100003&lng=en&nrm=iso$>$. access on  18  Oct.  2015.  http://dx.doi.org/10.1590/S0104-65002002000100003.

\item RODRIGUES, Bruno César; CRIPPA, Giulia. A recuperação da informação e o conceito de informação: o que é relevante em mediação cultural?. Perspect. ciênc. inf.,  Belo Horizonte ,  v. 16, n. 1, p. 45-64, Mar.  2011 .   Available from $<$http://www.scielo.br/scielo.php?script=sci\_arttext\&pid=S1413-99362011000100004\&lng=en\&nrm=iso$>$. access on  18  Oct.  2015.  http://dx.doi.org/10.1590/S1413-99362011000100004.
\end{enumerate}
